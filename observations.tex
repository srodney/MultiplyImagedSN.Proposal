%\begin{wrapfigure}{r}{0.3\textwidth}
%\minipage[r]{0.29\textwidth}
   
\begin{wraptable}{r}{2.25in}
 \caption{Detection Limits \label{tab:detectionLimits}}
 \begin{tabular}{ccc}
  \toprule
  \toprule
   WFC3/IR & Exp.Time & m$_{lim}$ $^{*}$ \\
   Filter  &  [min] &  [AB mag] \\
   \midrule
   F110W   &    12  &   26.3 \\
   F110W   &    20  &   26.6 \\
   F110W   &    30  &   26.9 \\
   F140W   &    12  &   25.9 \\
   F140W   &    20  &   26.2 \\ 
   F140W   &    30  &   26.5 \\
  \bottomrule
 \multicolumn{3}{p{2.6in}}{$^*$\,Apparent magnitude that yields S/N of 5 (optimum S/N$\sim$10) in the given exposure time.}
\end{tabular}
\end{wraptable}




%\end{minipage}
%\end{wrapfigure}

In Cycle 22, HST has just achieved a new capability for the discovery
of a multiply-imaged SN. Three key ingredients will make this program
feasible for the first time in this cycle: (1) the use of
WFC3-IR, (2) a SNAP survey strategy
with repeated shallow visits over many clusters, and (3) a carefully
optimized target list of clusters with both IR template imaging
and excellent mass models.

\noindent {\bf 1. WFC3-IR:~~~} Ground-based surveys and even HST/ACS programs
have looked for lensed
SN \citep[e.g.][]{Sharon:2007,Dawson:2009,Sharon:2010,Sand:2011}, but
none of these have had the capability to detect SN at $z\sim2$, even
with substantial magnification, so many multiply-imaged
galaxies were effectively unreachable to them.
\citet{Amanullah:2011} 
demonstrated the value of searching for high-$z$ SN at IR wavelengths,
where you sample rest-frame optical bands at the peak of the SN
SED. Using a K-band survey from the VLT with HAWK-I\footnote{VLT: Very
Large Telescope; HAWK-I: High Acuity Wide-field K-band Imager} they
discovered of a lens-magnified SN at $z\sim1.7$ behind the galaxy
cluster Abell 1689.  With the CANDELS and CLASH programs, we have
taken this IR search strategy above the atmosphere, showing that
WFC3-IR is capable of discovering even un-lensed SNe at
$z>1.5$ \citep{Rodney:2012,Jones:2013}, and finding three more
lens-magnified SNe \citep{Patel:2013,Nordin:2014}.  The ongoing Hubble
Frontier Fields (HFF) program (PI:Lotz) is gathering deep WFC3-IR
imaging of 6 galaxy clusters, and with our multi-cycle GO program
(PI:Rodney) we have already found 8 SN in the HFF data, including 2
with significant magnification.  However, all of these magnified SN
are too far away from the cluster center to be multiply-imaged. 

\noindent {\bf 2. Snapshots:~~~} A primary reason that
recent WFC3-IR surveys have not found any multiply-imaged SN is
because they are optimized for depth and wavelength coverage instead
of cadence.  CLASH, for example, collected WFC3-IR imaging of 25
clusters over 3 years, but many orbits were allocated to ACS and
WFC3-UVIS, and the time separation between the first and last IR image
on any single cluster was typically only $\sim$40 days.  Thus, in
practice each cluster only had one epoch suitable for a lensed SN
search, making any detection extremely unlikely.  The HFF program only
exacerbates this problem, by drilling even deeper on a much smaller
number of clusters.  However, CLASH, HFF and other programs have now
provided deep IR template imaging of many massive clusters from which
to construct difference images for SN discovery.  Our snapshot program
will capitalize on this rich archival treasury by delivering hundreds
of shallow visits across $\sim$25 clusters, covering a long time
baseline to maximize our chance of catching a multiply-imaged SN in
action.

\noindent {\bf 3. Cluster Target Selection:~~~} 
The HST archive now holds a deep trove of ACS and (most notably)
WFC3-IR imaging on strong-lensing galaxy clusters at redshifts
$z\sim0.2-0.7$.  This valuable data set has led to an explosion of
high quality cluster lens models, thanks to much effort in
supplemental observations and modeling \citep[e.g.][]{Kneib:2004,
Smith:2005, Limousin:2008, Bradac:2008, Richard:2009}.  Co-PI Zitrin
has had a leading role in this recent burst, principally through the
light-traces-mass (LTM) lens modeling technique, which particularly
excels with its predictive power for the discovery of multiply-imaged
galaxies \citep{Broadhurst:2005, Zitrin:2009a}.  Through the CLASH
program, this approach is being used to generate precise
mass models for 25 clusters, dramatically expanding the number of
well-studied strong-lensing clusters with many multiple-image
systems \citep[e.g.][]{Zitrin:2009a, Zitrin:2009b, Zitrin:2011a,
Zitrin:2011b, Zitrin:2011c, Merten:2011, Zitrin:2012a, Zitrin:2012b,
Zitrin:2013a, Zitrin:2013b, Coe:2012, Coe:2013}.  
The quality of our lens models will translate directly into the
uncertainty in our determination of \Ho.  We find that reaching
$\sim$10\% precision on \Ho\ is a very plausible benchmark, consistent
with  past estimates \citep{Bolton:2003,Oguri:2003,Riehm:2011}.

To build our target list, we have taken into account the trade-off
between the number of lensed galaxies and the length of their time
delays.  Very massive clusters will generally provide more
multiply-imaged background galaxies that could host a SN during our
survey.  However, very massive clusters will also on avarage yield
time delays which are too long to be of practical use hard to measure
on a reasonable time scale (\dt can be $\sim10^3$ years). In some
cases as few as $\sim10-20\%$ of the multiple-image pairs behind these
monster lenses would yield desirable time delays on the scale of
months to years.  Therefore, the optimal cluster lens targets are
medium-to-large lenses with Einstein radii of roughly
$\sim10-30\arcsec$.  There is also some dependence on the exact
structure and redshift of the lens, but as a rule of thumb such lenses
will each have about 5-20 multiple galaxy images with useful time
delays.

Table~\ref{tab:clusters} presents our list of 25 cluster targets,
which are spread across the sky to optimize snapshot
observability. The list is dominated by lenses of moderate strength,
but supplemented with 2-3 more massive clusters that contain of order
100 multiple images each.
