
Although this snapshot program is


time delay can be precisely defined even
without continuous monitor 
instead of a stochastically variable source (quasar) will make the
time delay measurement 





For a powerful complementary approach, our HST snapshot program will
open up a new chapter in time delay cosmography with the discovery of
a multiply-imaged SN behind a massive galaxy cluster.  

 the potential
to be 
delay measurements all suffer 


however, no multiply-imaged SN have yet been observed.  

Cosmography with time delays is a relatively young field, but is
quickly coming to maturity, with samples of $>$100 quasar time delays
possible within the next decade \textcolor{red}{(citation
  needed)}. 

This
narrowly focused discovery program is an extremely efficient way to
take a great leap forward in the relatively young field of time delay
cosmography.


 In
coming years, time delay measurements have the potential to provide
cosmographic precision that is on par with distance measurements from
Type Ia Supernovae (\SNIa) and baryon acoustic oscillations (BAO)
\citep{Linder:2011,Treu:2013}.  These ``low-redshift'' ($z<1000$) probes
provide a complement to the exceptional cosmological
precision afforded by the Cosmic Microwave Background (CMB), allowing
us to test dark energy models in the relatively recent eras of cosmic
acceleration ($z\lesssim 1$) and decleration ($z\gtrsim$ 1).  

Adding this new probe to the existing cosmographers
toolkit will provide two key improvements. 
First, time delay distances as a 
 very powerful checks
against systematic biases in SN and BAO distances. 



The seminal paper by \citet{Refsdal:1964} was the first to propose the
use of SN for measuring the time delay incited by a gravitational
lens.   This technique has been vigorously pursued 
