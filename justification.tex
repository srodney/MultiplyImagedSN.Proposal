
With this snapshot program, we propose to use HST's WFC3-IR detector
to discover the first multiply-imaged SN behind a strong lensing
galaxy cluster, setting the gold standard for future strong lensing
time delay measurements.  In recent years HST programs such as CLASH
and the Hubble Frontier Fields have built up a deep trove of WFC3-IR
imaging on strong-lensing galaxy clusters at redshifts $z\sim0.5$.
Many of these clusters have dozens of multiply-imaged background
galaxies at redshifts $1\lesssim z \lesssim 6$, which have been used
to produce well-constrained models of the cluster lensing potential.
When a SN inevitably appears within one of these multiply-imaged
galaxies, it will of course be multiply-imaged itself.  With that SN
we could measure a time delay distance with better than 6\%
precision, providing a unique and powerful
constraint on \Ho\ and dark energy.  With deep template imaging for
over 25 rich clusters in hand, {\bf \it we now have a viable pathway
to discover that first multiply-imaged SN with a small, single-cycle
snapshot program.}


\subsection{Time Delay Cosmography}

As light from a distant source passes through a galaxy cluster,
strong gravitational lensing causes multiple images to appear to the
observer, with a time separation between the images given by

\begin{equation}\label{eq:dt}
  \dt = \frac{\Dl \Ds}{\Dls} ( 1 + z_l ) \phi
\end{equation}

\noindent where \Dl, \Ds, and \Dls\ are angular diamater distances from 
the observer to the lens, observer to source, and lens to source,
respectively. The redshift of the lens is $z_l$, and $\phi$ is the
time delay potential, which includes a geometric component due to
light rays following different path lengths to the observer, plus a
general relativistic component due to differing values of the
gravitational potential along each path.

Each of the distances in Equation~\ref{eq:dt} carries a factor
of \Ho$^{-1}$, so if the lensing potential $\phi$ is well known, then a
time delay measurement provides a direct measurement of the Hubble
constant.  The distance ratio $\Dl\Ds/\Dls$ also has
unusual sensitivities to cosmological parameters as a function of
redshift that make this technique a particularly useful probe of
dynamic dark energy models \citep{Linder:2011}.

\citep{Refsdal:1964} first proposed the use of SN time delays as 
a means to measure \Ho.  Now 50 years later, we have only just begun
to realize the potential in this technique with the recent
measurement of a few dozen {\it quasar} time
delays \citep{Jackson:2007}, including just a handful with
particularly high precision \citep{Suyu:2010,Suyu:2013}.  These quasar
lenses generally suffer from a number of serious concerns, notably:
(1) the lensing potential (typically a single foreground galaxy) is
poorly constrained; (2) the time delays and angular separations are
quite small (tens of days, fractions of an arcsecond); and (3) the
source is continuously variable, requiring many years of stable
monitoring to resolve phase degeneracies.  Wide-field surveys in the
coming decade could deliver $>$100 quasar time delays, but these
problems represent unavoidable systematic biases for this sample.
 

This snapshot program will open the door for a much cleaner time delay
distance measurement with the discovery of a strongly lensed SN behind
a galaxy cluster.  We will target well-studied, massive clusters that
all have dozens of known multiple-image systems, meaning that the
lensing potentials $\phi$ are exquisitely well defined.  Typical time
delays through these clusters are months or years, allowing for more
precise measurement of \dt, with ample time to prepare for the
appearance of the second image.   A SN also provides
an inherently better time-delay source: the SN light curve has a
single peak, so there is no potential for phase ambiguity, and the age
of a SN relative to explosion can be precisely defined from light
curve shape and color, or from spectroscopic
cross-correlation \citep{Filippenko:1997,Blondin:2007} -- so the time
delay measurement does not require continuous long-term monitoring.  

Furthermore, with a cluster lens and SN source there is the
possibility of measuring independent distances to both components.
Cluster distances can be estimated using the Sunyaev-Zeldovich effect
and x-ray cluster luminosities \citep{Silk:1978}.  If the SN is of
Type Ia (a likely prospect), then light curve fitting can provide a
luminosity distance measurement with $\sim$8\%
precision \citep{Phillips:1993}.  A multiply-imaged \SNIa\ would also
be a unique chance to apply the distance-duality test: comparing the
luminosity distance $D_{s}^{(L)}$ inferred from light curve fitting
against the angular diameter distance $D_{s}^{(A)}$ derived from the
time delay.  If the ratio
$\eta_{DD} = D_{s}^{(L)} / ( D_{s}^{(A)}(1+z_s)^2 )$
deviates from unity, then this would signal systematic errors in one or
more distances, or a fundamental flaw in the concordance cosmology. 



